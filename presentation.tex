% vim: spelllang=de_20
\documentclass{beamer}

\title{Zellularautomaten: Datenstrukturen}
\author{Sebastian Ullrich}
\date{Sommersemester 2012}

\usepackage[utf8]{inputenc}
\usepackage[ngerman]{babel}

\usepackage{amsmath,amsfonts}
\usepackage{tikz}

\usetikzlibrary{arrows,positioning}

\begin{document}

\begin{frame}
    \titlepage
\end{frame}

\begin{frame}
    \tableofcontents
\end{frame}
\section{Motivation}
\begin{frame}{Was?}
    \begin{itemize}
        \item Eindimensionale Zellularautomaten eignen sich wunderbar dafür, lineare Datenstrukturen zu simulieren
        \item Wir betrachten die einfachsten davon: Stacks und Queues
            \begin{center}
                \begin{tikzpicture}[>=stealth]
                    \matrix[row sep=1cm,midway] {
                        \useasboundingbox (1,0) rectangle (6,1);
                        \draw (1,0) grid (5,1);
                        \draw (1,0) -- (6,0);
                        \draw (1,1) -- (6,1);
                        \node at (1.5,0.5) {a};
                        \node at (2.5,0.5) {b};
                        \node at (3.5,0.5) {c};
                        \\

                        \useasboundingbox (1,0) rectangle (6,1);
                        \useasboundingbox (5,1);
                        \draw (1,0) grid (5,1);
                        \draw (1,0) -- (6,0);
                        \draw (1,1) -- (6,1);
                        \node at (1.5,0.5) {z};
                        \node at (2.5,0.5) {a};
                        \node at (3.5,0.5) {b};
                        \node at (4.5,0.5) {c};
                        \draw[->,thick] (0.5,1.5) -- node[sloped,above] { $push$ $z$ } (1,1);
                        \\

                        \useasboundingbox (1,-1) rectangle (6,1);
                        \draw (1,0) grid (5,1);
                        \draw (1,0) -- (6,0);
                        \draw (1,1) -- (6,1);
                        \node at (1.5,0.5) {a};
                        \node at (2.5,0.5) {b};
                        \node at (3.5,0.5) {c};
                        \draw[->,thick] (1,0) -- node[sloped,below] { $pop$ } (0.5,-0.5);
                        \\
                    };
                \end{tikzpicture}
            \end{center}
        \item Ganz so einfach in der Implementierung dann doch nicht\dots
    \end{itemize}
\end{frame}

\begin{frame}{Warum?}
    \begin{itemize}
        \item Regel 110 ist Turing-vollständig, aber kaum zu überblicken
        \item Wir bauen uns eine einfachere Turing-Maschine
            \begin{center}
                \begin{tikzpicture}[scale=0.5,node distance=2mm]
                    \draw (-5,0) -- (7,0);
                    \draw (-5,1) -- (7,1);
                    \draw (-0.5,0) -- (-0.5,1);
                    \draw (2.5,0) -- (2.5,1);
                    \node (1) at (1,0.5) { $Q_T \times \Sigma$ };
                    \node[left=of 1] { $\leftarrow$ Stack 1 };
                    \node[right=of 1] { Stack 2 $\rightarrow$ };
                \end{tikzpicture}
            \end{center}
            ($\Rightarrow$ \emph{Zipper}-Datenstruktur)
        \item Simulierung der Übergangsfunktion:
            \begin{itemize}
                \item $\delta(q,a) = (q',a',N)$:
                    \begin{center}
                        \begin{tikzpicture}[scale=0.5,node distance=2mm]
                            \draw (-5,0) -- (7,0);
                            \draw (-5,1) -- (7,1);
                            \draw (-0.5,0) -- (-0.5,1);
                            \draw (2.5,0) -- (2.5,1);
                            \node (1) at (1,0.5) { $(q',a')$ };
                            \node[left=of 1] { $nop$ };
                            \node[right=of 1] { $nop$ };
                        \end{tikzpicture}
                    \end{center}
                \item $\delta(q,a) = (q',a',L)$:
                    \begin{center}
                        \begin{tikzpicture}[scale=0.5,node distance=2mm]
                            \draw (-5,0) -- (7,0);
                            \draw (-5,1) -- (7,1);
                            \draw (-0.5,0) -- (-0.5,1);
                            \draw (2.5,0) -- (2.5,1);
                            \node (1) at (1,0.5) { $(q',a_L)$ };
                            \node[left=of 1] { $a_L := pop$ };
                            \node[right=of 1] { $push$ $a'$ };
                        \end{tikzpicture}
                    \end{center}
                \item $\delta(q,a) = (q',a',R)$:
                    \begin{center}
                        \begin{tikzpicture}[scale=0.5,node distance=2mm]
                            \draw (-5,0) -- (7,0);
                            \draw (-5,1) -- (7,1);
                            \draw (-0.5,0) -- (-0.5,1);
                            \draw (2.5,0) -- (2.5,1);
                            \node (1) at (1,0.5) { $(q',a_R)$ };
                            \node[left=of 1] { $push$ $a'$ };
                            \node[right=of 1] { $a_R := pop$ };
                        \end{tikzpicture}
                    \end{center}
            \end{itemize}
        \item (Demo)
    \end{itemize}
\end{frame}

\section{Implementierung eines Stacks ohne Zeitverlust}
\begin{frame}[fragile]{Wie?}
    \begin{itemize}
        \item Hintereinander aufgereiht können die Symbole augenscheinlich nicht schnell genug aus dem Stack herausgeholt werden

            $\Rightarrow$ Wir speichern mehrere Symbole in einer Zelle
    \begin{tikzpicture}[every filldraw/.style=lightgray]
        \matrix[row sep=1cm,column sep=1cm] {
            \draw[scale=0.4] (1) (0,0) grid (4,3);
            \begin{scope}[scale=0.4,xshift=12,yshift=12]
                \node[anchor=east] at (-1,1) { $nop$ };
                \node at (0,2) {a};
                \node at (0,1) {b};
                \node at (1,2) {c};
                \node at (1,1) {d};
            \end{scope}
            &


            \filldraw[scale=0.4,lightgray] (0,2) rectangle (1,1);
            \draw[scale=0.4] (1) (0,0) grid (4,3);
            \begin{scope}[scale=0.4,xshift=12,yshift=12]
                \node[anchor=east] at (-1,1) { $pop$ };
                \node at (0,2) {b};
                \node at (1,2) {c};
                \node at (1,1) {d};
            \end{scope}
            \\

            \filldraw[scale=0.4,lightgray] (1,2) rectangle (2,1);
            \draw[scale=0.4] (1) (0,0) grid (4,3);
            \begin{scope}[scale=0.4,xshift=12,yshift=12]
                \node[anchor=east] at (-1,1) { $nop$ };
                \node at (0,2) {b};
                \node at (0,1) {c};
                \node at (1,2) {d};
            \end{scope}
            &

            \filldraw[scale=0.4,lightgray] (0,2) rectangle (1,3);
            \draw[scale=0.4] (1) (0,0) grid (4,3);
            \begin{scope}[scale=0.4,xshift=12,yshift=12]
                \node[anchor=east] at (-1,1) { $push$ $a$ };
                \node at (0,2) {a};
                \node at (0,1) {b};
                \node at (0,0) {c};
                \node at (1,2) {d};
            \end{scope}
            \\

            \filldraw[scale=0.4,lightgray] (0,2) rectangle (2,3);
            \draw[scale=0.4] (1) (0,0) grid (4,3);
            \begin{scope}[scale=0.4,xshift=12,yshift=12]
                \node[anchor=east] at (-1,1) { $pop$ };
                \node at (0,2) {b};
                \node at (1,2) {c};
                \node at (1,1) {d};
            \end{scope}
            &

            \filldraw[scale=0.4,lightgray] (1,1) rectangle (2,2);
            \draw[scale=0.4] (1) (0,0) grid (4,3);
            \begin{scope}[scale=0.4,xshift=12,yshift=12]
                \node[anchor=east] at (-1,1) { $nop$ };
                \node at (0,2) {b};
                \node at (0,1) {c};
                \node at (1,2) {d};
            \end{scope}
            \\
        };
    \end{tikzpicture}
    \end{itemize}
\end{frame}

\begin{frame}{Wie? - formalisiert}
    \begin{itemize}
        \item Der Stack besteht aus drei \emph{Bändern}, jede Zelle aus drei \emph{Registern}
        \item Wir wollen jeweils die ersten zwei Register gefüllt halten
    \item Eine Zelle mit drei vollen Registern gibt den Inhalt ihres dritten Registers an ihren rechten Nachbarn ab, der ihn seinen Registern voranstellt
        \begin{center}
            \begin{tikzpicture}[scale=0.4,>=stealth]
                \filldraw[lightgray] (0,3) rectangle (1,0);
                \filldraw[lightgray] (1,3) rectangle (2,1);
                \draw (0,0) grid (2,3);
                \begin{scope}[shorten >=0.5mm]
                    \draw[->,thick] (0.5,0.5) -- (1.5,2.5);
                    \draw[->,darkgray] (1.5,2.5) -- (1.5,1.5);
                    \draw[->,darkgray] (1.5,1.5) -- (1.5,0.5);
                \end{scope}
            \end{tikzpicture}
        \end{center}
    \item Eine Zelle mit weniger als zwei vollen Registern stiehlt das erste Register ihres rechten Nachbarn
        \begin{center}
            \begin{tikzpicture}[scale=0.4,>=stealth]
                \filldraw[lightgray] (0,3) rectangle (1,2);
                \filldraw[lightgray] (1,3) rectangle (2,1);
                \draw (0,0) grid (2,3);
                \begin{scope}[shorten >=0.5mm]
                    \draw[->,thick] (1.5,2.5) -- (0.5,1.5);
                    \draw[->,darkgray] (1.5,1.5) -- (1.5,2.5);
                \end{scope}
            \end{tikzpicture}
        \end{center}
    \end{itemize}
\end{frame}
\end{document}
