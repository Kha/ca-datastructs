\documentclass{article}

\usepackage[utf8]{inputenc}
\usepackage[ngerman]{babel}

\usepackage{amsmath,amsfonts}
\usepackage[hyperref,standard,thref]{ntheorem}
\usepackage{hyperref}

\DeclareMathOperator{\bild}{Bild}

\begin{document}

\newcommand{\ca}{\ensuremath\mathcal{A}}
\newcommand{\abs}[1]{\lvert#1\rvert}
\newcommand{\ceil}[1]{\lceil#1\rceil}
\newcommand{\zrange}[1]{\mathbb{G}_{#1}}

\section{Foo}

\begin{definition}
    Wir nennen eine Familie von Zellularautomaten $\ca_\Sigma$ \emph{$k$-Band-Automaten}, falls für die jeweilige Zustandsmenge $Q_\Sigma$ gilt:
    \begin{equation}
        \abs{Q_\Sigma} \in \mathcal{O}(\abs{\Sigma}^k) \label{def:ntape}
    \end{equation}
\end{definition}

\begin{satz}
    \label{thm:tape-hom}
    Jeder $k$-Band-Automat ist homomorph zu einem Automat mit der Zustandsmenge $\zrange{\ceil{c}} \times \Sigma^n$, wobei $c$ die Konstante aus dem $\mathcal{O}$-Kalkül in \eqref{def:ntape} ist.

    \begin{beweis}
        Die Aussage folgt wegen $\abs{Q_\Sigma} \leq c \abs{\Sigma}^k \leq \ceil{c} \abs{\Sigma}^k$ aus Mächtigkeitsargumenten.
    \end{beweis}
\end{satz}

\begin{definition}
    Für eine zelluläre Datenstruktur mit einer Operation \emph{pop} ohne Zeitverlust sei $\Gamma_t(n) \subseteq \Sigma^*$ die Buchstabenfolge, die bei wiederholtem Aufruf von \emph{pop} ab dem Zeitpunkt $t$ zurückgegeben wird, wenn die ersten $n$ Zellen als eigener Automat betrachtet werden.

    Für Automaten mit Zustandsmenge $\Sigma^k$ ist dies normalerweise die Konkatenation der Inhalt der ersten $n$ Zellen.
    Für den Gesamtinhalt setzen wir noch $\Gamma_t := \Gamma_t(\infty)$.
\end{definition}

Für das Ergebnis $a \in \Sigma$ einer \emph{pop}-Operation ist $a$ Präfix von $\Gamma_t$ und, da \emph{pop} ohne Zeitverlust agiert, sogar von $\Gamma_t(2)$. Aus der Signalgeschwindigkeit folgt allgemein:
\begin{korollar}
    \label{thm:gamma-in-gamma}
    Für $w \in \Sigma^*$ Präfix von $\Gamma_t$ ist w schon Präfix von $\Gamma_t(\abs{w}+1)$.
\end{korollar}

\begin{satz}
    \label{cor:spacesize}
    Für $k$-Band-Automaten gibt es für alle $n \in \mathbb{N}, m > kn$ ein Alphabet $\Sigma$ und ein Wort $w \in \Sigma^m$, nach dessen Pushen gilt:
    \[ \abs{\Gamma_t(n)} \leq kn \]
    \begin{beweis}
        Definiere die zu $\Gamma$ gehörige Abbildung
        \[ \gamma_n : Q^n \rightarrow \Sigma^*, \gamma_n(c_t(1), \dots, c_t(n)) = \Gamma_t(n)\]
        Angenommen, es gälte für ein $n$ und $m$ und alle Alphabete $\Sigma$ und Konfigurationen $\abs{\Gamma_t(n)} > kn$, also $\abs{\bild(\gamma_n)} \geq \abs{\Sigma}^{kn+1}$.
        Es folgt
        \[ \abs\Sigma^{kn+1} < \abs{\bild{\gamma_n}} \leq \abs{Q^n} \leq (\ceil{c}\abs{\Sigma}^k)^n = \abs{\Sigma}^{kn} \ceil{c}^n \]
        also
        \[ kn + 1 \leq kn + n \log_{\abs\Sigma}(\ceil{c}) \]
        Für $\abs{\Sigma}$ groß genug ergibt sich der Widerspruch.
    \end{beweis}
\end{satz}

Im Folgenden seien $\Sigma$ und $w$ für alle Push-Operationen entsprechend gewählt.

\begin{korollar}
    \label{cor:signalpush}
    Aus der Signalgeschwindigkeit in Zellularautomaten folgt: Wird in Schritt $t_0+1$ ein Zeichen $a \in \Sigma$ gepusht, gilt für $t > t_0$:
    \begin{equation}
        a \in \Gamma_t(t-t_0) \label{cor:signal}
    \end{equation}
\end{korollar}

\begin{lemma}
    \label{lem:2tape-halffilled}
    Für 2-Band-Automaten gilt nach Pushen eines Zeichens zum Zeitpunkt $t_0 + 1$, falls keine weiteren Aktionen folgen:
    \[ \abs{\Gamma_{t_0+m}(n)} = n \quad\forall\, 1 \leq n \leq \abs{\Gamma_{t_0}},m > n \]
    Anschaulich: Das zweite Band ist danach ungefüllt.
    \begin{beweis}
        Wir nehmen an, es gäbe $m$ und $n$, sodass $\abs{\Gamma_{t_0+m}(n)} \geq n$. Betrachten wir die alternativen Konfigurationen $c'$, die durch Pushen von $w$, $\abs{w}=n$ Zeichen ab dem Zeitpunkt $t_0+(m-n)+1$ entstehen würden, so gilt wegen der Signalgeschwindigkeit $c_{t_0+m}(n') = c'_{t_0+m}(n') \;\forall n' > n$, also ist $\Gamma_{t_0+m}(n)$ Suffix von $\Gamma'_{t_0+m}(n)$, mit \thref{cor:signalpush} insgesamt $\Gamma'_{t_0+m}(n) = w \cdot \Gamma_{t_0+m}(n)$. Es folgt $\abs{\Gamma'_{t_0+m}(n)} = n + \abs{\Gamma_{t_0+m}(n)} > n + n$, ein Widerspruch zu \thref{cor:spacesize}.
    \end{beweis}
\end{lemma}

\begin{satz}
    Es gibt keinen 2-Band-Automaton mit \emph{pop}-Operation ohne Zeitverlust.
    \begin{beweis}
        Betrachte nach Einfügen der Elemente $a, b, c, d$ 4 \emph{pop}-Operationen ab $t_0+1$. Um die ersten zwei Operationen zu bedienen, muss $a \in \Gamma_{t_0+1}(2)$ und $b \in \Gamma_{t_0+2}(2)$ gelten, wegen der Signalgeschwindigkeit sogar schon $b \in \Gamma_{t_0+1}(2)$. Analog folgt aus $c \in \Gamma_{t_0+3}(2) c \in \Gamma_{t_0+1}(3)$. Es folgt aus \thref{lem:2tape-halffilled} $d \notin \Gamma_{t_0+1}(3)$ und wiederum aus Geschwindigkeitsgründen $d \notin \Gamma_{t_0+4}(2)$. Die vierte Operation kann also nicht bedient werden.
    \end{beweis}
\end{satz}

\end{document}
