% vim: spelllang=de_20
\documentclass{article}

\title{Datenstrukturen in Zellularautomaten}
\author{Sebastian Ullrich}
\date{Sommersemester 2012}

\usepackage[utf8]{inputenc}
\usepackage[T1]{fontenc}
\usepackage[ngerman]{babel}
\usepackage{xspace}

\usepackage{amsmath,amsfonts}
\usepackage[hyperref,standard,thref]{ntheorem}
\usepackage{hyperref}
\usepackage{tikz}

\usetikzlibrary{matrix,arrows,positioning}

\DeclareMathOperator{\zip}{zip}

\begin{document}

\newcommand{\ca}{\ensuremath\mathcal{A}}
\newcommand{\abs}[1]{\lvert#1\rvert}
\newcommand{\ceil}[1]{\lceil#1\rceil}
\newcommand{\zrange}[1]{\{1,\dots,#1\}}
\newcommand{\bild}[1]{\text{Bild}(#1)}
\newcommand{\N}{\mathbb{N}}
\newcommand{\ve}[3]{\begin{pmatrix}#1\\#2\\#3\end{pmatrix}}
\newcommand{\vet}[2]{\begin{pmatrix}#1\\#2\end{pmatrix}}

\newcommand{\pop}{\ensuremath{\mathit{pop}}\xspace}
\newcommand{\popZiel}{\ensuremath{\mathit{popZiel}}\xspace}
\newcommand{\popQuelle}{\ensuremath{\mathit{popQuelle}}\xspace}
\newcommand{\nop}{\ensuremath{\mathit{nop}}\xspace}
\newcommand{\push}{\ensuremath{\mathit{push}}\xspace}
\newcommand{\pushZiel}{\ensuremath{\mathit{pushZiel}}\xspace}
\newcommand{\pushQuelle}{\ensuremath{\mathit{pushQuelle}}\xspace}
\newcommand{\enqueue}{\ensuremath{\mathit{enqueue}}\xspace}
\newcommand{\dequeue}{\ensuremath{\mathit{dequeue}}\xspace}
\newcommand{\shiftUp}{\ensuremath{\mathit{shiftUp}}\xspace}
\newcommand{\demogrid}[1]{
    \begin{scope}
        \clip (-0.1,-0.1) rectangle (3.5,#1+0.1);
        \draw (1) (0,0) grid (4,#1);
    \end{scope}
}

\maketitle
\newpage
\mbox{}
\newpage

\begin{abstract}
	Um die Mächtigkeit des Berechnungsmodells der Zellularautomaten zu zeigen, bieten sich Reduktionen anderer Modelle auf dieses an. Dafür ist es oft nötig, eine Datenstruktur wie den Stack eines Kellerspeicherautomaten oder die Bänder einer Turing-Maschine auf einen Zellularautomaten abzubilden. Martin Kutrib beschreibt in \cite{kutrib08} die Implementierung eines Stacks und einer Queue ohne Zeitverlust. Diese Arbeit baut darauf auf und erweitert sie um eine formale Konstruktion der zwei Zellularautomaten als auch eine abstrakte Definition eines \emph{Stack}-Zellularautomaten, mithilfe welcher Kutribs Implementierung auf Korrektheit überprüft werden kann.
\end{abstract}

\section{Formale Definition}

\begin{figure}[h]
    \centering
    \begin{tikzpicture}[scale=0.5,>=stealth]
        \draw (1,0) grid (5,1);
        \draw (1,0) -- (6,0);
        \draw (1,1) -- (6,1);
        \foreach \n in {1,...,4} { \node at (\n+0.5,0.5) {\n}; }
        \node at (5.5,0.5) {\dots};

        \node (source) at (-0.5,2.5) { $\Sigma$ };
        \node (sink) at (-0.5,-1.5) { $\Sigma$ };
        \draw[->,thick] (source) -- node[right=0.3,above] { $\delta_{\push}$ } (1,1);
        \draw[->,thick] (1,0) -- node[below right] { $\gamma$ } (sink);
    \end{tikzpicture}
    \caption{Schematischer Aufbau und Verhalten einer Datenstruktur}
\end{figure}
Wir betrachten Datenstrukturen mit Von-Neumann-Nachbarschaft von Radius 1 auf dem Gitter $R = \N$, das wir als nach rechts offenes Band darstellen. Für Zellen ab Position 2 existiert also eine Überführungsfunktion $\delta : Q^3 \rightarrow Q$, die den neuen Zustand einer Zelle in Abhängigkeit des Zustands ihres linken Nachbarn, der Zelle selbst und ihres rechten Nachbarn bestimmt.

Die erste Zelle dagegen besitzt keinen linken Nachbarn, kann allerdings Befehle von außerhalb des Automaten (bildlich "`von links"') entgegennehmen, welche wir bei den betrachteten Datenstrukturen Stack und Queue beide Male \push und \pop nennen wollen. Zusammen mit einer Operation \nop, der Nulloperation, können wir damit das Verhalten der ersten Zelle und gleichzeitig das von außen sichtbare Verhalten der ganzen Datenstruktur durch drei Überführungsfunktionen
\begin{align*}
    \delta_{\push} &: \Sigma \times Q^2 \rightarrow Q \\
    \delta_{\pop}, \delta_{\nop} &: Q^2 \rightarrow Q
\end{align*}
und eine \emph{Inhaltsfunktion}
\[ \gamma : Q \rightarrow \Sigma \cup \{\epsilon\} \]
die das oberste Symbol oder $\epsilon$ bei einer leeren Struktur zurückgibt, beschreiben. Dabei ist das Ein-/Ausgabealphabet $\Sigma$ von der letztendlichen Anwendung der Datenstruktur abhängig.

Auf Basis dieser lokalen Überführungsfunktionen seien entsprechende globale Überführungsfunktionen definiert:
\begin{align*}
    \Delta_{\push} &: \Sigma \times Q^\N \rightarrow Q^\N \\
    \Delta_{\pop} &: Q^\N \rightarrow (\Sigma \cup \{\epsilon\}) \times Q^\N \\
    \Delta_{\nop} &: Q^\N \rightarrow Q^\N
\end{align*}
Beispielhaft lässt sich $\Delta_\nop$ definieren durch
\[ \Delta_\nop(c) := \left(\gamma(c(1)), \left< i \mapsto \begin{cases}
        \delta_\nop(c(1), c(2)) &\text{falls } i = 1 \\
        \delta(c(i-1), c(i), c(i+1)) &\text{sonst }
\end{cases} \right> \right) \]

Da wir aber auch Datenstrukturen betrachten wollen, die mit Zeitverlust agieren, definieren wir darüber hinaus
\begin{align*}
    \Delta^*_{\push} &: \Sigma \times Q^\N \rightarrow Q^\N,& (a,c) &\mapsto \Delta_{\push}(a,\Delta^i_{\nop}(c)) \\
    \Delta^*_{\pop} &: Q^\N \rightarrow (\Sigma \cup \{\epsilon\}) \times Q^\N,& c &\mapsto \Delta_{\pop}(\Delta^i_{\nop}(c))
\end{align*}
Die Zahl $i$ der notwendigen Nulloperationen vor der eigentlichen Operation ist dabei von der Implementierung der Datenstruktur, möglicherweise sogar von der Konfiguration $c$ abhängig.

Natürlich wollen wir noch Bedingungen an das Verhalten einer Datenstruktur stellen. Dazu müssen wir aber Aussagen über den \emph{Inhalt} der Datenstruktur machen können:
\begin{definition}
    Für eine globale Konfiguration $c \in Q^\N$ sei $\Gamma(c) \in \Sigma^*$ die Folge von Zeichen, die bei wiederholtem Aufruf von \pop zurückgegeben werden:
    \[ \Gamma(c) :=
        \begin{cases}
            \epsilon &\text{falls } \Delta^*_{\pop}(c) \in \{\epsilon\} \times Q^\N \\
            a \cdot \Gamma(c') &\text{falls }\Delta^*_{\pop}(c)=:(a,c') \in \Sigma \times Q^\N
    \end{cases} \]
\end{definition}

Nun lassen sich die erwarteten Eigenschaften einer Implementierung formal definieren:
\begin{definition}
    Ein \emph{Stack}-Zellularautomat ist ein Tupel $(Q, \delta, \delta_{\push}, \delta_{\pop}, \delta_{\nop}, q_0)$, das folgende Axiome erfüllt: \\
    Für jede \emph{valide} Konfiguration $c$, also eine aus $c_0 \equiv q_0$ durch beliebig wiederholte Anwendung von $\Delta_\nop$, $\Delta^*_\push$ und $\Delta^*_\pop$ ableitbare Konfiguration, muss gelten:
    \begin{enumerate}
        \item $\delta(q_0, q_0, q_0) = q_0$: $\,q_0$ ist der Ruhezustand
        \item $\forall a \in \Sigma: \Gamma(\Delta^*_{\push}(a,c)) = a \cdot \Gamma(c)$
        \item $\forall a \in \Sigma, w \in \Sigma^*: \left[ \Gamma(c) = a \cdot w \Rightarrow \Delta^*_{\pop}(c) = (a,c'), \Gamma(c') = w \right]$
        \item $\Gamma(c) = \epsilon \Rightarrow \Delta^*_{\pop}(c) = (\epsilon,c'), \Gamma(c') = \epsilon$
        \item $\Gamma(\Delta_{\nop}(c)) = \Gamma(c)$
    \end{enumerate}
    Insbesondere arbeitet der Stack \emph{ohne Zeitverlust}, falls die Axiome schon durch $\Delta_{\push}$ und $\Delta_{\pop}$ erfüllt werden.
\end{definition}

Analog kann eine Queue-Struktur formal definiert werden. Für alle folgenden Implementierungen der zwei Datenstrukturen wurden die Axiome mithilfe von \emph{QuickCheck} (\cite{hughes00}) auf zufälligen Konfigurationen erfolgreich überprüft.

\begin{beispiel}
    \begin{figure}[h]
        \centering
        \begin{tikzpicture}[scale=0.5,node distance=2mm]
            \draw (-5,0) -- (7,0);
            \draw (-5,1) -- (7,1);
            \draw (-0.5,0) -- (-0.5,1);
            \draw (2.5,0) -- (2.5,1);
            \node (1) at (1,0.5) { $Q_T \times \Sigma$ };
            \node[left=of 1] { $\leftarrow$ Stack 1 };
            \node[right=of 1] { Stack 2 $\rightarrow$ };
        \end{tikzpicture}
        \caption{Simulation einer Turing-Maschine durch zwei Stacks}
    \end{figure}
    Gegeben eine Stack-Implementierung $(Q_S, \delta_S, \delta_{\push}, \delta_{\pop}, \delta_{\nop}, q^S_0)$ ohne Zeitverlust, können wir mit zwei solcher Stacks eine Turing-Maschine $(Q_T, \Sigma, \delta_T, q^T_0, \sqcup, q_f)$ ohne Zeitverlust simulieren (das Bandalphabet sei der Einfachheit halber das Eingabealphabet). Dazu speichern wir in Zelle 0 des Gitters $\mathbb{Z}$ den aktuellen Zustand und das Symbol unter dem Lesekopf und links und rechts davon in je einem der Stacks das linke bzw. rechte Restband. Es ergibt sich $Q := (Q_T \times \Sigma) \cup Q_S$ und die Überführungsfunktion für Zelle 0
    \[ \delta_0(q_{-1}, (q, a), q_1) :=
        \begin{cases}
            (q', a') &\text{für } \delta_T(q, a) = (q', a', N) \\
            (q', \bar\gamma(q_{-1})) &\text{für } \delta_T(q, a) = (q', a', L) \\
            (q', \bar\gamma(q_1)) &\text{für } \delta_T(q, a) = (q', a', R) \\
    \end{cases} \]
    mit
    \[ \bar\gamma : Q_S \rightarrow \Sigma, q \mapsto \begin{cases}
            \gamma(q) &\text{für } \gamma(q) \in \Sigma \\
            \sqcup &\text{für } \gamma(q) = \epsilon
    \end{cases} \]
    sowie für Zelle 1 (und analog für Zelle -1)
    \[ \delta_1((q, a), q_1, q_2) :=
        \begin{cases}
            \delta_{\nop}(q_1, q_2) &\text{für } \delta_T(q, a) = (q', a', N) \\
            \delta_{\push}(a', q_1, q_2) &\text{für } \delta_T(q, a) = (q', a', L) \\
            \delta_{\pop}(q_1, q_2) &\text{für } \delta_T(q, a) = (q', a', R) \\
    \end{cases} \]
\end{beispiel}

\newpage
\section{Zwei Stack-Implementierungen}

\subsection{Ein 3-Band-Stack ohne Zeitverlust}

Kutribs Stack ohne Zeitverlust in \cite{kutrib08} speichert bis zu drei Symbole in \emph{Registern} einer Zelle und wird von Kutrib deshalb als \emph{3-Band-Automat} bezeichnet. Es bietet sich damit die Zustandsmenge $(\Sigma \cup \{\epsilon\})^3$ mit $q_0 = (\epsilon,\epsilon,\epsilon)$ an.

\begin{figure}[h]
    \centering
    \begin{tikzpicture}[every filldraw/.style=lightgray,every node/.style={anchor=base}]
        \matrix (mym) [row sep=1cm,column sep=1cm,every cell/.style={scale=0.4}] {
            \demogrid{3}
            \begin{scope}[xshift=12,yshift=5]
                \node[anchor=east,yshift=3] at (-1,1) { 1. \nop };
                \node at (0,2) {a};
                \node at (0,1) {b};
                \node at (1,2) {c};
                \node at (1,1) {d};
            \end{scope}
            &


            \filldraw[lightgray] (0,2) rectangle (1,1);
            \demogrid{3}
            \begin{scope}[xshift=12,yshift=5]
                \node[anchor=east,yshift=3] at (-1,1) { 2. \pop };
                \node at (0,2) {b};
                \node at (1,2) {c};
                \node at (1,1) {d};
            \end{scope}
            \\

            \filldraw[lightgray] (1,2) rectangle (2,1);
            \demogrid{3}
            \begin{scope}[xshift=12,yshift=5]
                \node[anchor=east,yshift=3] at (-1,1) { 3. \nop };
                \node at (0,2) {b};
                \node at (0,1) {c};
                \node at (1,2) {d};
            \end{scope}
            &

            \filldraw[lightgray] (0,2) rectangle (1,3);
            \demogrid{3}
            \begin{scope}[xshift=12,yshift=5]
                \node[anchor=east,yshift=3] at (-1,1) { 4. {\push} $a$ };
                \node at (0,2) {a};
                \node at (0,1) {b};
                \node at (0,0) {c};
                \node at (1,2) {d};
            \end{scope}
            \\

            \filldraw[lightgray] (0,1) rectangle (2,2);
            \demogrid{3}
            \begin{scope}[xshift=12,yshift=5]
                \node[anchor=east,yshift=3] at (-1,1) { 5. \pop };
                \node at (0,2) {b};
                \node at (1,2) {c};
                \node at (1,1) {d};
            \end{scope}
            &

            \filldraw[lightgray] (1,1) rectangle (2,2);
            \demogrid{3}
            \begin{scope}[xshift=12,yshift=5]
                \node[anchor=east,yshift=3] at (-1,1) { 6. \nop };
                \node at (0,2) {b};
                \node at (0,1) {c};
                \node at (1,2) {d};
            \end{scope}
            \\
        };
    \end{tikzpicture}
    \caption{Beispielablauf des 3-Band-Stacks. Graue Zellen beschreiben die Einfügungen bzw. Löschungen durch eine Operation}
\end{figure}
Der Inhalt eines solchen Automaten ist die Konkatenation seiner Zellinhalte:
\begin{align*}
    \gamma(\ve{r_1}{r_2}{r_3}) &:= r_1 \\
    \Gamma(c) &= \prod^\infty_{i=1} c(i)
\end{align*}
Diese Konkatenations-Produkt ist endlich, da sich in jeder validen, also in endlicher Zeit aus dem leeren Band entstandenen Konfiguration nur endlich viele Zellen von $q_0$ unterscheiden können.

Jede Zelle des Automaten versucht, genau zwei ihrer Register gefüllt zu halten (wobei in jedem Zustand die vollen Register vor den leeren stehen werden). Für die Überführungsfunktion ergeben sich daraus folgende Regeln:
\begin{enumerate}
    \item Eine Zelle mit drei vollen Registern gibt den Inhalt ihres dritten Registers an ihren rechten Nachbarn ab, der ihn seinen Registern voranstellt.
        \begin{center}
            \begin{tikzpicture}[scale=0.4,>=stealth]
                \filldraw[lightgray] (0,3) rectangle (1,0);
                \filldraw[lightgray] (1,3) rectangle (2,1);
                \draw (0,0) grid (2,3);
                \begin{scope}[shorten >=0.5mm]
                    \draw[->,thick] (0.5,0.5) -- (1.5,2.5);
                    \draw[->,darkgray] (1.5,2.5) -- (1.5,1.5);
                    \draw[->,darkgray] (1.5,1.5) -- (1.5,0.5);
                \end{scope}
            \end{tikzpicture}
        \end{center}
    \item Eine Zelle mit weniger als zwei vollen Registern stiehlt das erste Register ihres rechten Nachbarn, falls gefüllt.
        \begin{center}
            \begin{tikzpicture}[scale=0.4,>=stealth]
                \filldraw[lightgray] (0,3) rectangle (1,2);
                \filldraw[lightgray] (1,3) rectangle (2,1);
                \draw (0,0) grid (2,3);
                \begin{scope}[shorten >=0.5mm]
                    \draw[->,thick] (1.5,2.5) -- (0.5,1.5);
                    \draw[->,darkgray] (1.5,1.5) -- (1.5,2.5);
                \end{scope}
            \end{tikzpicture}
        \end{center}
\end{enumerate}

Betrachten wir beide Regeln jeweils aus Sicht der \emph{Quell-} und \emph{Zielzelle}, ergeben sich die vier Überführungsfunktionen
\begin{align*}
    \delta_{\pushQuelle}(l,\ve{q_1}{q_2}{q_3},r) &= \ve{q_1}{q_2}{\epsilon} \\
    \delta_{\pushZiel}(\ve{l_1}{l_2}{l_3},\ve{q_1}{q_2}{q_3},r) &= \ve{l_3}{q_1}{q_2} \text{ falls } l_3 \in \Sigma \\
    \delta_{\popQuelle}(\ve{l_1}{\epsilon}{\epsilon},\ve{q_1}{q_2}{q_3},r) &= \ve{q_2}{q_3}{\epsilon} \\
    \delta_{\popZiel}(l,\ve{q_1}{\epsilon}{\epsilon},\ve{r_1}{r_2}{r_3}) &= \ve{q_1}{r_1}{\epsilon}
\end{align*}
In allen nichtspezifizierten Fällen sei der Zellzustand unverändert.

Da in einem Zeitschritt gegebenenfalls mehrere Fälle auf die gleiche Zelle zutreffen können, müssen diese vier Teilüberführungsfunktionen noch in der richtigen Reihenfolge zusammengesetzt werden. Die Definition einer \emph{Komposition} von Überführungsfunktionen über der gleichen Zustandsmenge $Q$ ist noch recht naheliegend: Es sei $\Delta_Q := Q^{Q \times Q \times Q}$ die Menge dieser Überführungsfunktionen (bei Von-Neumann-Nachbarschaft mit Radius 1) und darauf definiert
\begin{align*}
    \circ : \Delta_Q \times \Delta_Q &\rightarrow \Delta_Q \\
    (\delta_1 \circ \delta_2)(l,q,r) &:= \delta_1(l,\delta_2(l,q,r),r)
\end{align*}
Schwieriger ist dagegen die Festlegung der Reihenfolge. Es ergeben sich folgende Einschränkungen:
\begin{enumerate}
    \item $\pushZiel \leftarrow \pushQuelle$: Wurde gerade ein Zeichen in die Zelle gepusht und dadurch alle Register gefüllt, darf \pushQuelle das letzte Register nicht im gleichen Zeitschritt wieder leeren.
    \item $\popQuelle \leftarrow \popZiel$: Wurde durch \popQuelle die Zahl der gefüllten Zellen von zwei auf eines verringert, darf \popZiel im gleichen Zeitschritt nicht schon ein Zeichen von rechts übernehmen, da für den rechten Nachbarn die Zelle noch nicht unterfüllt erscheint.
    \item $\popQuelle \leftarrow \pushQuelle$: Sieht der rechte Nachbar eine voll gefüllte Zelle und wird das dritte Register übernehmen, darf \popQuelle dessen Inhalt nicht ins zweite Register schieben.
    \item $\pushZiel \leftarrow \popZiel$: Sieht der rechte Nachbar, dass nur eine Zelle gefüllt ist, und wird sein erstes Register löschen, muss \popZiel dieses übernehmen, bevor \pushZiel die Zahl der gefüllten Register ggf. auf 2 erhöht.
\end{enumerate}

\begin{figure}[h]
    \centering
    \begin{tikzpicture}[>=stealth]
        \node (pushZiel) { \pushZiel };
        \node (popQuelle) [below=of pushZiel] { \popQuelle };
        \node (pushQuelle) [right=of pushZiel] { \pushQuelle };
        \node (popZiel) [right=of popQuelle] { \popZiel };
        \draw[->] (pushQuelle) edge (pushZiel) edge (popQuelle)
        (popZiel) edge (pushZiel) edge (popQuelle);
    \end{tikzpicture}
    \caption{Abhängigkeitsdiagramm der vier Überführungsfunktionen}
\end{figure}
Damit ergibt sich als eine von vier Möglichkeiten beispielhaft
\[ \delta := \delta_{\pushZiel} \circ \delta_{\popQuelle} \circ \delta_{\popZiel} \circ \delta_{\pushQuelle} \]

Es verbleiben die Überführungsfunktionen des Stackkopfes. Diese können in diesem Fall allerdings leicht aus $\delta$ abgeleitet werden, indem die jeweilige Operation als virtuelle Zelle links vom Kopf betrachtet wird, wobei $b \in \Sigma$ ein beliebiges Symbol zum Auffüllen der Tupel ist.
\begin{align*}
    \delta_{\push}(a, q_1, q_2) &:= \delta(\ve{b}{b}{a}, q_1, q_2) \\
    \delta_{\pop}(q_1, q_2) &:= \delta(\ve{\epsilon}{\epsilon}{\epsilon}, q_1, q_2) \\
    \delta_{\nop}(q_1, q_2) &:= \delta(\ve{b}{b}{\epsilon}, q_1, q_2) \\
\end{align*}

\subsection{Ein 2-Band-Stack mit Zeitverlust}

Exakt analog kann auch ein Stack mit nur zwei Bändern entworfen werden. Dieser hält wieder das letzte Band leer, also nur das erste Band gefüllt, und die vier Teilüberführungsfunktionen ergeben sich aus den angepassten Regeln
\begin{enumerate}
    \item Eine Zelle mit zwei vollen Registern gibt den Inhalt ihres zweiten Registers an ihren rechten Nachbarn ab, der ihn seinen Registern voranstellt.
        \begin{center}
            \begin{tikzpicture}[scale=0.4,>=stealth]
                \filldraw[lightgray] (0,2) rectangle (1,0);
                \filldraw[lightgray] (1,2) rectangle (2,1);
                \draw (0,0) grid (2,2);
                \begin{scope}[shorten >=0.5mm]
                    \draw[->,thick] (0.5,0.5) -- (1.5,1.5);
                    \draw[->,darkgray] (1.5,1.5) -- (1.5,0.5);
                \end{scope}
            \end{tikzpicture}
        \end{center}
    \item Eine Zelle mit zwei leeren Registern stiehlt das erste Register ihres rechten Nachbarn, falls gefüllt.
        \begin{center}
            \begin{tikzpicture}[scale=0.4,>=stealth]
                \filldraw[lightgray] (1,2) rectangle (2,1);
                \draw (0,0) grid (2,2);
                \begin{scope}[shorten >=0.5mm]
                    \draw[->,thick] (1.5,1.5) -- (0.5,1.5);
                \end{scope}
            \end{tikzpicture}
        \end{center}
\end{enumerate}
Da die Symbole im Ruhezustand nun allerdings linear im ersten Band angeordnet sind, kann die \pop-Operation nicht mehr ohne Zeitverlust bedient werden: $\Delta^*_{\pop}=\Delta_{\pop} \circ \Delta_{\nop}$.
\begin{figure}[h]
    \centering
    \begin{tikzpicture}[every node/.style={anchor=base}]
        \matrix [row sep=1cm,column sep=1cm] {
            \draw[scale=0.4] (0,0) grid (4,2);
            \begin{scope}[scale=0.4,xshift=12,yshift=5]
                \node[anchor=east,yshift=4] at (-1,0.5) { 1. \nop };
                \node at (0,1) {a};
                \node at (1,1) {b};
                \node at (2,1) {c};
            \end{scope}
            &


            \filldraw[scale=0.4,lightgray] (0,2) rectangle (1,1);
            \draw[scale=0.4] (0,0) grid (4,2);
            \begin{scope}[scale=0.4,xshift=12,yshift=5]
                \node[anchor=east,yshift=4] at (-1,0.5) { 2. \pop };
                \node at (1,1) {b};
                \node at (2,1) {c};
            \end{scope}
            \\

            \filldraw[scale=0.4,lightgray] (1,2) rectangle (2,1);
            \draw[scale=0.4] (0,0) grid (4,2);
            \begin{scope}[scale=0.4,xshift=12,yshift=5]
                \node[anchor=east,yshift=4] at (-1,0.5) { 3. \nop };
                \node at (0,1) {b};
                \node at (2,1) {c};
            \end{scope}
            &

            \filldraw[scale=0.4,lightgray] (0,2) rectangle (1,1);
            \filldraw[scale=0.4,lightgray] (2,2) rectangle (3,1);
            \draw[scale=0.4] (0,0) grid (4,2);
            \begin{scope}[scale=0.4,xshift=12,yshift=5]
                \node[anchor=east,yshift=4] at (-1,0.5) { 4. \pop };
                \node at (1,1) {c};
            \end{scope}
            \\
        };
    \end{tikzpicture}
    \caption{Beispielablauf des 2-Band-Stacks}
\end{figure}


\newpage
\section{Eine Queue-Implementierung ohne Zeitverlust}

Auch für Queues beschreibt Kutrib eine Implementierung ohne Zeitverlust, wiederum mit drei Bändern. In dieser werden wieder die ersten zwei Bänder gefüllt gehalten, um \dequeue-Operationen genau wie \pop-Operationen beim 3-Band-Stack ohne Zeitverlust anbieten zu können. Das dritte Band ist damit noch frei und kann die Eingabe der \enqueue-Operation von der Kopfzelle vorbei am Inhalt der ersten zwei Bänder ans Schlangenende weiterleiten.
\begin{figure}[h]
    \centering
    \begin{tikzpicture}[every filldraw/.style=lightgray,every node/.style={anchor=base}]
        \matrix (mym) [row sep=1cm,column sep=1cm,every cell/.style={scale=0.4}] {
            \demogrid{3}
            \begin{scope}[xshift=12,yshift=5]
                \node[anchor=east,yshift=3] at (-1,1) { 1. \nop };
                \node at (0,2) {a};
                \node at (0,1) {b};
                \node at (1,2) {c};
                \node at (1,1) {d};
                \node at (2,2) {e};
            \end{scope}
            &

            \filldraw[lightgray] (0,2) rectangle (1,1);
            \demogrid{3}
            \begin{scope}[xshift=12,yshift=5]
                \node[anchor=east,yshift=3] at (-1,1) { 2. \dequeue };
                \node at (0,2) {b};
                \node at (1,2) {c};
                \node at (1,1) {d};
                \node at (2,2) {e};
            \end{scope}
            &

            \filldraw[lightgray] (1,2) rectangle (2,1);
            \demogrid{3}
            \begin{scope}[xshift=12,yshift=5]
                \node[anchor=east,yshift=3] at (-1,1) { 3. \nop };
                \node at (0,2) {b};
                \node at (0,1) {c};
                \node at (1,2) {d};
                \node at (2,2) {e};
            \end{scope}
            \\

            \filldraw[lightgray] (2,2) rectangle (3,3);
            \demogrid{3}
            \begin{scope}[xshift=12,yshift=5]
                \node[anchor=east,yshift=3] at (-1,1) { 4. \nop };
                \node at (0,2) {b};
                \node at (0,1) {c};
                \node at (1,2) {d};
                \node at (1,1) {e};
            \end{scope}
            &

            \filldraw[lightgray] (0,0) rectangle (1,1);
            \demogrid{3}
            \begin{scope}[xshift=12,yshift=5]
                \node[anchor=east,yshift=3] at (-1,1) { 5. \enqueue f };
                \node at (0,2) {b};
                \node at (0,1) {c};
                \node at (1,2) {d};
                \node at (1,1) {e};
                \node at (0,0) {f};
            \end{scope}
            &

            \filldraw[lightgray] (1,0) rectangle (2,1);
            \demogrid{3}
            \begin{scope}[xshift=12,yshift=5]
                \node[anchor=east,yshift=3] at (-1,1) { 6. \nop };
                \node at (0,2) {b};
                \node at (0,1) {c};
                \node at (1,2) {d};
                \node at (1,1) {e};
                \node at (1,0) {f};
            \end{scope}
            \\
            \demogrid{3}
            \begin{scope}[xshift=12,yshift=5]
                \node[anchor=east,yshift=3] at (-1,1) { 7. \nop };
                \node at (0,2) {b};
                \node at (0,1) {c};
                \node at (1,2) {d};
                \node at (1,1) {e};
                \node at (2,2) {f};
            \end{scope}
            \\
        };
    \end{tikzpicture}
    \caption{Beispielablauf der 3-Band-Queue. Die ersten vier Schritte sind äquivalent zum Verhalten des 3-Band-Stacks.}
\end{figure}

Da das Verhalten der ersten zwei Bänder mit dem des 3-Band-Stacks übereinstimmt, lässt sich für diese leicht eine Teilüberführungsfunktion
\[ \delta_\dequeue \in \Delta_{(\Sigma \cup \{\epsilon\})^2} \]
angeben, indem man die Überführungsfunktion des 3-Band-Stacks auf zwei Bänder einschränkt.  Die Bewegung nach rechts des dritten Bandes $\delta_\enqueue \in \Delta_{(\Sigma \cup \{\epsilon\})}$ ist ebenfalls trivial:
\[ \delta_\enqueue(l,q,r) := l \]

Um die beiden Teilüberführungsfunktionen zusammenzufügen, bietet sich eine Faltungsfunktion der folgenden Art an:
\begin{align*}
    \zip : \Delta_{Q_1} \times \Delta_{Q_2} &\rightarrow \Delta_{Q_1 \times Q_2} \\
    \zip(\delta_1, \delta_2)(\vet{l_1}{l_2},\vet{q_1}{q_2},\vet{r_1}{r_2}) &:= \vet{\delta_1(l_1,q_1,r_1)}{\delta_2(l_2,q_2,r_2)}
\end{align*}

Es fehlt nur noch die Interaktion der zwei Teile:
\[ \delta_\shiftUp \in \Delta_{(\Sigma \cup \{\epsilon\})^3} \]
\begin{align*}
    \delta_\shiftUp(l,\ve{\epsilon}{\epsilon}{q_3},r) &:= \ve{q_3}{\epsilon}{\epsilon} \\
    \delta_\shiftUp(l,\ve{q_1}{\epsilon}{q_3},\ve{\epsilon}{\epsilon}{\epsilon}) &:= \ve{q_1}{q_3}{\epsilon} \text{ für } q_1 \in \Sigma
\end{align*}

Damit setzt sich die Gesamtüberführungsfunktion zusammen als
\[ \delta := \delta_\shiftUp \circ \zip(\delta_\dequeue,\delta_\enqueue) \]
\begin{figure}[h]
    \centering
    \begin{tikzpicture}[scale=0.4,every node/.style={anchor=base}]
        \begin{scope}
            \clip (-0.1,-0.1) rectangle (4.5,3.1);
            \draw[color=gray] (1) (0,0) grid (5,3);
        \end{scope}
        \draw[->,>=stealth,thick] (3.5,2.5) -- (2.5,1.5) -- (2.5,2.5) -- (1.5,1.5) -- (1.5,2.5) -- node[above=0.2] { \dequeue } (0.5,1.5) -- (0.5,2.5) -- (-0.5,1.5);
        \draw[->,>=stealth,thick] (-0.5,0.5) -- node[below=0.1] { \enqueue } (3.3,0.5);
        \draw[->,>=stealth,thick] (3.5,0.5) -- node[below=0.2,sloped] { \shiftUp } (3.5,2.3);
    \end{tikzpicture}
    \caption{Schematische Darstellung der Teilübergangsfunktion der 3-Band-Queue}
\end{figure}

\newpage
\begin{thebibliography}{9}
        \bibitem{kutrib08}
        M. Kutrib: \emph{Cellular Automata – A Computational Point of View}, Studies in Computational Intelligence (SCI) 113, 183–227 (2008)

        \bibitem{hughes00}
        Koen Claessen, John Hughes: \emph{QuickCheck: a lightweight tool for random testing of Haskell programs}, ICFP 2000, 268-279
\end{thebibliography}
\end{document}
