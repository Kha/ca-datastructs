\documentclass{article}

\usepackage[utf8]{inputenc}
\usepackage[ngerman]{babel}

\usepackage{amsmath,amsfonts}
\usepackage[hyperref,standard,thref]{ntheorem}
\usepackage{hyperref}

\DeclareMathOperator{\bild}{Bild}

\begin{document}

\newcommand{\ca}{\ensuremath\mathcal{A}}
\newcommand{\abs}[1]{\lvert#1\rvert}
\newcommand{\ceil}[1]{\lceil#1\rceil}
\newcommand{\zrange}[1]{\mathbb{G}_{#1}}

\begin{definition}
    Wir nennen eine Familie von Zellularautomaten $\ca_\Sigma$ \emph{$k$-Band-Automaten}, falls für die jeweilige Zustandsmenge $Q_\Sigma$ gilt:
    \begin{equation}
        \abs{Q_\Sigma} \in \mathcal{O}(\abs{\Sigma}^k) \label{def:ntape}
    \end{equation}
\end{definition}

Die tatsächliche Form der Zustandsmenge des Automaten kann beliebig weit von $k$-Tupeln entfernt sein, aber \eqref{def:ntape} sichert zu, dass wir sie stets auf unsere gewohnte Definition eines $k$-Bandes zurückführen können:

\begin{satz}
    \label{thm:tape-hom}
    Jeder $k$-Band-Automat lässt sich in einen Automaten mit der Zustandsmenge $\zrange{\ceil\alpha} \times \Sigma^k$ überführen, der an seinen Operationen gemessen gleich arbeitet. Dabei ist $\alpha$ die Konstante aus dem $\mathcal{O}$-Kalkül in \eqref{def:ntape}.

    \begin{beweis}
        Die Aussage folgt wegen $\abs{Q_\Sigma} \leq \alpha \abs{\Sigma}^k \leq \ceil\alpha \abs{\Sigma}^k$ aus Mächtigkeitsargumenten.
    \end{beweis}
\end{satz}

Dieser Satz sagt allerdings nichts aus über die tatsächliche Speicherung der Inhalte im Zellraum, insbesondere muss ein gepushtes Zeichen $a \in \Sigma$ nicht in dieser Form überhaupt auf einem Band stehen. Um den Inhalt zu inspizieren, muss also auf die externen Operationen zurückgegriffen werden:

\begin{definition}
    Für eine zelluläre Datenstruktur mit einer Operation \emph{pop} ohne Zeitverlust sei in Konfiguration $c$ $\Gamma(c) \subseteq \Sigma^*$ ihr \emph{Inhalt}, also die Buchstabenfolge, die bei wiederholtem Aufruf von \emph{pop} zurückgegeben wird.

    Den Inhalt der ersten $n$ Zellen definieren wir als
        \[ \Gamma(c,n) := \Gamma(c) \big|_m \]
    wobei $\Gamma(c) \big|_m$ den Präfix von $\Gamma(c)$ der Länge $m$ bezeichnet und $m$ definiert ist als
    \[ m := \max \{m' \in \mathbb{N} \mid \Gamma(c) \big|_{m'} = \Gamma(c') \big|_{m'} \;\forall c' \text{ valide Konfiguration}, c' \big|_n = c \big|_n \} \]
    also $\Gamma(c,n)$ längster gemeinsamer Inhaltspräfix aller Konfigurationen mit den gleichen ersten $n$ Zellen ist oder anders gesagt der Inhalt, der von allen Zellen $> n$ unabhängig ist.

    Für Automaten mit Zustandsmenge $\Sigma^k$ ist dies normalerweise die Konkatenation des Inhalts der ersten $n$ Zellen.
\end{definition}

\begin{satz}
    \label{thm:locontent}
    Nach Pushen von $2n-3$ Zeichen gibt es einen Zeitpunkt $t_0$, sodass $\abs{\Gamma(c,n)} \geq 2n-3$.
    \begin{beweis}
        Welchen Präfix von $\Gamma(c)$ kann eine Konfiguration $c'$ mit $c' \big|_n = c \big|_n$ verändern? Das Signal des ersten Pops erreicht $c'(n+1) \neq c(n+1)$ zum Zeitpunkt $t_0+n$. Damit könnte $c'^{t_0+n-1+n-2}(2)$ beeinflusst werden, also die $2n-3$-te Pop-Operation.
    \end{beweis}
\end{satz}

\begin{satz}
    \label{cor:spacesize}
    Für $k$-Band-Automaten gibt es für alle $n \in \mathbb{N}, l > kn$ ein Alphabet $\Sigma$ und ein Wort $w \in \Sigma^l$, nach dessen Pushen gilt:
    \[ \abs{\Gamma(c,n)} \leq kn \]
    \begin{beweis}
        Angenommen, es gälte für ein $n$ und $l$ und alle Alphabete $\Sigma$ und Konfigurationen $c$ $\abs{\Gamma(c,n)} > kn$, also $\abs{\bild(\Gamma(\cdot,n))} \geq \abs{\Sigma}^{kn+1}$.
        Es folgt
        \[ \abs\Sigma^{kn+1} < \abs{\bild(\Gamma(\cdot,n))} \leq \abs{Q^n} \leq (\ceil{c}\abs{\Sigma}^k)^n = \abs{\Sigma}^{kn} \ceil{c}^n \]
        also
        \[ kn + 1 \leq kn + n \log_{\abs\Sigma}(\ceil{c}) \]
        Für $\abs{\Sigma}$ groß genug ergibt sich der Widerspruch.
    \end{beweis}
\end{satz}

\begin{satz}
    Es gibt keinen 2-Band-Automaton mit \emph{pop}-Operation ohne Zeitverlust.
    \begin{beweis}
        Wähle $w \in \Sigma^{2n+1}$ nach \thref{cor:spacesize}. Nach Pushen von $w \big|_{2n-3}$ gibt es nach \thref{thm:locontent} $t_0$, sodass $\abs{\Gamma(c,n)} \geq 2n-3$. Pushe dann die letzten vier Buchstaben und es gilt, da $c(n+1)$ nicht beeinflusst worden sein kann, $\abs{\Gamma(c,n)} \geq 2n+1$, im Widerspruch zu \thref{cor:spacesize}.
    \end{beweis}
\end{satz}

\end{document}
