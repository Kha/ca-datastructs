% vim: spelllang=de_20
\documentclass{article}

\title{Zellularautomaten: Datenstrukturen}
\author{Sebastian Ullrich}
\date{Sommersemester 2012}

\usepackage[utf8]{inputenc}
\usepackage[ngerman]{babel}

\usepackage{amsmath,amsfonts}
\usepackage[hyperref,standard,thref]{ntheorem}
\usepackage{hyperref}
\usepackage{tikz}

\usetikzlibrary{matrix,arrows,positioning}

\begin{document}

\newcommand{\ca}{\ensuremath\mathcal{A}}
\newcommand{\abs}[1]{\lvert#1\rvert}
\newcommand{\ceil}[1]{\lceil#1\rceil}
\newcommand{\zrange}[1]{\mathbb{G}_{#1}}
\newcommand{\bild}[1]{\text{Bild}(#1)}
\newcommand{\N}{\mathbb{N}}
\newcommand{\ve}[3]{\begin{pmatrix}#1\\#2\\#3\end{pmatrix}}

\newcommand{\pop}{\ensuremath{\mathit{pop}}}
\newcommand{\popZiel}{\ensuremath{\mathit{popZiel}}}
\newcommand{\popQuelle}{\ensuremath{\mathit{popQuelle}}}
\newcommand{\nop}{\ensuremath{\mathit{nop}}}
\newcommand{\push}{\ensuremath{\mathit{push}}}
\newcommand{\pushZiel}{\ensuremath{\mathit{pushZiel}}}
\newcommand{\pushQuelle}{\ensuremath{\mathit{pushQuelle}}}

\maketitle
\newpage

\begin{abstract}
    Um die Mächtigkeit des Berechnungsmodells der Zellularautomaten zu zeigen, bieten sich Reduktionen anderer Modelle auf dieses an. Dafür ist es oft nötig, eine Datenstruktur wie den Stack eines Kellerspeicherautomaten oder die Bänder einer Turing-Maschine auf einen Zellularautomaten abzubilden. Martin Kutrib beschreibt in \cite{kutrib08} die Implementierung eines Stacks und einer Queue ohne Zeitverlust. Diese Arbeit baut darauf auf und erweitert sie um eine allgemeine, formale Definition der zwei Datenstrukturen, wodurch beweisbare Aussagen über ihre Eigenschaften geschlossen werden können. Insbesondere wird bewiesen, dass Stacks erst ab einer gewissen Größe der Zustandsmenge ohne Zeitverlust implementiert werden können und damit Kutribs Implementierung in dieser Hinsicht als minimal angesehen werden kann.
\end{abstract}

\section{Definition}

\begin{figure}[h]
    \centering
    \begin{tikzpicture}[scale=0.5,>=stealth]
        \draw (1,0) grid (5,1);
        \draw (1,0) -- (6,0);
        \draw (1,1) -- (6,1);
        \foreach \n in {1,...,4} { \node at (\n+0.5,0.5) {\n}; }
        \node at (5.5,0.5) {\dots};

        \node (source) at (-0.5,2.5) { $\Sigma$ };
        \node (sink) at (-0.5,-1.5) { $\Sigma$ };
        \draw[->,thick] (source) -- node[right=0.3,above] { $\delta_{\push}$ } (1,1);
        \draw[->,thick] (1,0) -- node[below right] { $\gamma$ } (sink);
    \end{tikzpicture}
    \caption{Schematischer Aufbau und Verhalten einer Datenstruktur}
\end{figure}
Wir betrachten Datenstrukturen mit Von-Neumann-Nachbarschaft von Radius 1 auf dem Gitter $R = \N$, das wir als nach rechts offenes Band darstellen. Für Zellen ab Position 2 existiert also eine Übergangsfunktion $\delta : Q^3 \rightarrow Q$, die den neuen Zustand einer Zelle in Abhängigkeit des Zustands ihres linken Nachbarn, der Zelle selbst und ihres rechten Nachbarn bestimmt.

Die erste Zelle dagegen besitzt keinen linken Nachbarn, kann allerdings Befehle von außerhalb des Automaten (bildlich "`von links"') entgegennehmen, welche wir bei den betrachteten Datenstrukturen Stack und Queue beide Male \push und \pop nennen wollen. Zusammen mit einer Operation \nop, der Nulloperation, können wir damit das Verhalten der ersten Zelle und gleichzeitig das von außen sichtbare Verhalten der ganzen Datenstruktur durch drei Übergangsfunktionen
\begin{align*}
    \delta_{\push} &: \Sigma \times Q^2 \rightarrow Q \\
    \delta_{\pop}, \delta_{\nop} &: Q^2 \rightarrow Q
\end{align*}
und eine \emph{Inhaltsfunktion}
\[ \gamma : Q \rightarrow \Sigma \cup {\epsilon} \]
die das oberste Symbol oder $\epsilon$ bei einer leeren Struktur zurückgibt, beschreiben. Dabei ist das Alphabet $\Sigma$ von der letztendlichen Anwendung der Datenstruktur abhängig.

Auf Basis der vier $\delta$ seien entsprechend noch definiert
\begin{align*}
    \Delta_{\push} &: \Sigma \times Q^\N \rightarrow Q^\N \\
    \Delta_{\pop} &: Q^\N \rightarrow (\Sigma \cup \{\epsilon\}) \times Q^\N \\
    \Delta_{\nop} &: Q^\N \rightarrow Q^\N
\end{align*}
Da wir aber auch Datenstrukturen betrachten wollen, die mit Zeitverlust agieren, definieren wir noch
\begin{align*}
    \Delta^*_{\push} &: \Sigma \times Q^\N \rightarrow Q^\N,& (a,c) &\mapsto \Delta_{\push}(a,\Delta^i_{\nop}(c)) \\
    \Delta^*_{\pop} &: Q^\N \rightarrow (\Sigma \cup \{\epsilon\}) \times Q^\N,& c &\mapsto \Delta_{\pop}(\Delta^i_{\nop}(c))
\end{align*}
Die Zahl $i$ der notwendigen Nulloperationen vor der eigentlichen Operation ist dabei von der Implementierung der Datenstruktur, möglicherweise sogar von der Konfiguration $c$ abhängig.

Natürlich wollen wir noch Bedingungen an das Verhalten einer Datenstruktur stellen. Dazu müssen wir aber Aussagen über den \emph{Inhalt} der Datenstruktur machen können:
\begin{definition}
    Für eine globale Konfiguration $c \in Q^\N$ sei $\Gamma(c) \in \Sigma^*$ die Folge von Zeichen, die bei wiederholtem Aufruf von \pop zurückgegeben werden:
    \[ \Gamma(c) := \begin{cases}
            \epsilon &\text{falls } \Delta^*_{\pop}(c) \in \{\epsilon\} \times Q^\N \\
        a \cdot \Gamma(c') &\text{falls }\Delta^*_{\pop}(c)=:(a,c') \in \Sigma \times Q^\N
    \end{cases} \]
\end{definition}

Nun lassen sich die erwarteten Eigenschaften einer Implementierung endlich formal definieren:
\begin{definition}
    Wir nennen einen Zellularautomaten $(Q, \delta, \delta_{\push}, \delta_{\pop}, \delta_{\nop}, q_0)$ einen Stack, falls er folgende Axiome erfüllt:

    Für jede \emph{valide} Konfiguration $c$, also eine aus $c_0 \equiv q_0$ durch beliebig wiederholte Anwendung der drei $\Delta$ ableitbare Konfiguration, muss gelten:
    \begin{enumerate}
        \item $\forall a \in \Sigma: \Gamma(\Delta^*_{\push}(a,c)) = a \cdot \Gamma(c)$
        \item $\forall a \in \Sigma, w \in \Sigma^*: \left[ \Gamma(c) = a \cdot w \Rightarrow \Delta^*_{\pop}(c) = (a,c'), \Gamma(c') = w \right]$
        \item $\Gamma(c) = \epsilon \Rightarrow \Delta^*_{\pop}(c) = (\epsilon,c'), \Gamma(c') = \epsilon$
        \item $\Gamma(\Delta_{\nop}(c)) = \Gamma(c)$
    \end{enumerate}
    Insbesondere arbeitet der Stack \emph{ohne Zeitverlust}, falls die Axiome schon durch $\Delta_{\push}$ und $\Delta_{\pop}$ erfüllt werden.
\end{definition}

Analog kann eine Queue-Struktur formal definiert werden.

\begin{beispiel}
    \begin{figure}[h]
        \centering
        \begin{tikzpicture}[scale=0.5,node distance=2mm]
            \draw (-5,0) -- (7,0);
            \draw (-5,1) -- (7,1);
            \draw (-0.5,0) -- (-0.5,1);
            \draw (2.5,0) -- (2.5,1);
            \node (1) at (1,0.5) { $Q_T \times \Sigma$ };
            \node[left=of 1] { $\leftarrow$ Stack 1 };
            \node[right=of 1] { Stack 2 $\rightarrow$ };
        \end{tikzpicture}
        \caption{Simulation einer Turing-Maschine durch zwei Stacks}
    \end{figure}
    Gegeben eine Stack-Implementierung $(Q_S, \delta_S, \delta_{\push}, \delta_{\pop}, \delta_{\nop}, q^S_0)$ ohne Zeitverlust, können wir mit zwei Stacks eine Turing-Maschine $(Q_T, \Sigma, \delta_T, q^T_0, q_f)$ ohne Zeitverlust simulieren (das Bandalphabet sei der Einfachheit halber das Eingabealphabet). Dazu speichern wir in Zelle 0  den aktuellen Zustand und das Symbol unter dem Lesekopf und links und rechts davon in je einem der Stacks das linke bzw. rechte Restband. Es ergibt sich $Q := (Q_T \times \Sigma) \cup Q^2_S$ und die Übergangsfunktion für Zelle 0
    \[ \delta_0(q_{-1}, (q, a), q_1) :=
    \begin{cases}
        (q', a') &\text{für } \delta_T(q, a) = (q', a', N) \\
        (q', \bar\gamma(q_{-1})) &\text{für } \delta_T(q, a) = (q', a', L) \\
        (q', \bar\gamma(q_1)) &\text{für } \delta_T(q, a) = (q', a', R) \\
    \end{cases} \]
mit
\[ \bar\gamma : Q_S \rightarrow \Sigma, q \mapsto \begin{cases}
        \gamma(q) &\text{für } \gamma(q) \in \Sigma \\
        \sqcup &\text{für } \gamma(q) = \epsilon
    \end{cases} \]
und Zelle 1
    \[ \delta_1((q, a), q_1, q_2) :=
    \begin{cases}
        \delta_{\nop}(q_1, q_2) &\text{für } \delta_T(q, a) = (q', a', N) \\
        \delta_{\push}(a', q_1, q_2) &\text{für } \delta_T(q, a) = (q', a', L) \\
        \delta_{\pop}(q_1, q_2) &\text{für } \delta_T(q, a) = (q', a', R) \\
    \end{cases} \]
und analog für Zelle -1.
\end{beispiel}

\section{Stack: Implementierungen}

\subsection{Ein 3-Band-Stack ohne Zeitverlust}

Kutribs Stack ohne Zeitverlust in \cite{kutrib08} besitzt die Zustandsmenge $(\Sigma \cup \{\epsilon\})^3$ mit $q_0 = (\epsilon,\epsilon,\epsilon)$, was er einen \emph{3-Band-Automaten} nennt, die drei Werte des Tupels \emph{Register}.

Der Inhalt eines solchen Automaten ist die Konkatenation seiner Zellinhalte:
\begin{align*}
    \gamma(\ve{r_1}{r_2}{r_3}) &:= r_1 \\
\Gamma(c) &= \sum^\infty_{i=1} c(i)
\end{align*}
Diese Summe ist endlich, da sich in jeder validen, also in endlicher Zeit aus dem leeren Band entstandenen Konfiguration nur endlich viele Zellen von $q_0$ unterscheiden können.

Jede Zelle des Automaten versucht, genau zwei ihrer Register gefüllt zu halten (wobei in jedem Zustand die vollen Register vor den leeren stehen werden). Für die Übergangsfunktion folgen folgende Regeln:
\begin{enumerate}
    \item Eine Zelle mit drei vollen Registern gibt den Inhalt ihres dritten Registers an ihren rechten Nachbarn ab, der ihn seinen Registern voranstellt.
        \begin{center}
            \begin{tikzpicture}[scale=0.4,>=stealth]
                \filldraw[lightgray] (0,3) rectangle (1,0);
                \filldraw[lightgray] (1,3) rectangle (2,1);
                \draw (0,0) grid (2,3);
                \begin{scope}[shorten >=0.5mm]
                    \draw[->,thick] (0.5,0.5) -- (1.5,2.5);
                    \draw[->,darkgray] (1.5,2.5) -- (1.5,1.5);
                    \draw[->,darkgray] (1.5,1.5) -- (1.5,0.5);
                \end{scope}
            \end{tikzpicture}
        \end{center}
    \item Eine Zelle mit weniger als zwei vollen Registern stiehlt das erste Register ihres rechten Nachbarn, falls gefüllt.
        \begin{center}
            \begin{tikzpicture}[scale=0.4,>=stealth]
                \filldraw[lightgray] (0,3) rectangle (1,2);
                \filldraw[lightgray] (1,3) rectangle (2,1);
                \draw (0,0) grid (2,3);
                \begin{scope}[shorten >=0.5mm]
                    \draw[->,thick] (1.5,2.5) -- (0.5,1.5);
                    \draw[->,darkgray] (1.5,1.5) -- (1.5,2.5);
                \end{scope}
            \end{tikzpicture}
        \end{center}
\end{enumerate}

\begin{figure}[h]
    \centering
    \begin{tikzpicture}[every filldraw/.style=lightgray,every node/.style={anchor=base}]
        \matrix (mym) [row sep=1cm,column sep=1cm] {
            \draw[scale=0.4] (1) (0,0) grid (4,3);
            \begin{scope}[scale=0.4,xshift=12,yshift=5]
                \node[anchor=east,yshift=3] at (-1,1) { 1. \nop };
                \node at (0,2) {a};
                \node at (0,1) {b};
                \node at (1,2) {c};
                \node at (1,1) {d};
            \end{scope}
            &


            \filldraw[scale=0.4,lightgray] (0,2) rectangle (1,1);
            \draw[scale=0.4] (1) (0,0) grid (4,3);
            \begin{scope}[scale=0.4,xshift=12,yshift=5]
                \node[anchor=east,yshift=3] at (-1,1) { 2. \pop };
                \node at (0,2) {b};
                \node at (1,2) {c};
                \node at (1,1) {d};
            \end{scope}
            \\

            \filldraw[scale=0.4,lightgray] (1,2) rectangle (2,1);
            \draw[scale=0.4] (1) (0,0) grid (4,3);
            \begin{scope}[scale=0.4,xshift=12,yshift=5]
                \node[anchor=east,yshift=3] at (-1,1) { 3. \nop };
                \node at (0,2) {b};
                \node at (0,1) {c};
                \node at (1,2) {d};
            \end{scope}
            &

            \filldraw[scale=0.4,lightgray] (0,2) rectangle (1,3);
            \draw[scale=0.4] (1) (0,0) grid (4,3);
            \begin{scope}[scale=0.4,xshift=12,yshift=5]
                \node[anchor=east,yshift=3] at (-1,1) { 4. \push $a$ };
                \node at (0,2) {a};
                \node at (0,1) {b};
                \node at (0,0) {c};
                \node at (1,2) {d};
            \end{scope}
            \\

            \filldraw[scale=0.4,lightgray] (0,1) rectangle (2,2);
            \draw[scale=0.4] (1) (0,0) grid (4,3);
            \begin{scope}[scale=0.4,xshift=12,yshift=5]
                \node[anchor=east,yshift=3] at (-1,1) { 5. \pop };
                \node at (0,2) {b};
                \node at (1,2) {c};
                \node at (1,1) {d};
            \end{scope}
            &

            \filldraw[scale=0.4,lightgray] (1,1) rectangle (2,2);
            \draw[scale=0.4] (1) (0,0) grid (4,3);
            \begin{scope}[scale=0.4,xshift=12,yshift=5]
                \node[anchor=east,yshift=3] at (-1,1) { 6. \nop };
                \node at (0,2) {b};
                \node at (0,1) {c};
                \node at (1,2) {d};
            \end{scope}
            \\
        };
    \end{tikzpicture}
    \caption{Beispielablauf des 3-Band-Stacks. Graue Zellen beschreiben die Einfügungen bzw. Löschungen durch eine Operation}
\end{figure}
Betrachten wir beide Regeln jeweils aus Sicht der \emph{Quell-} und \emph{Zielzelle}, ergeben sich die vier Übergangsfunktionen
\begin{align*}
    \delta_{\pushQuelle}(q_1,\ve{r_1}{r_2}{r_3},q_3) &= \ve{r_1}{r_2}{\epsilon} \\
    %\delta_{\pushQuelle}(q_1,q_2,q_3) &= (q_{21}, q_{22}, \epsilon) \\
    \delta_{\pushZiel}(\ve{r_1}{r_2}{r_3},\ve{s_1}{s_2}{s_3},q_3) &= \ve{r_3}{s_1}{s_2} \text{ falls } r_3 \in \Sigma \\
    \delta_{\popQuelle}(\ve{r_1}{\epsilon}{\epsilon},\ve{s_1}{s_2}{s_3},q_3) &= \ve{s_2}{s_3}{\epsilon} \text { falls } s_1 \in \Sigma \\
    \delta_{\popZiel}(q_1,\ve{r_1}{\epsilon}{\epsilon},\ve{s_1}{s_2}{s_3}) &= \begin{cases}
        \ve{r_1}{s_1}{\epsilon} &\text { falls } s_1,r_1 \in \Sigma \\
        \ve{s_1}{\epsilon}{\epsilon} &\text { falls } r_1 \in \Sigma, r_1 = \epsilon
    \end{cases}
\end{align*}
In allen nichtspezifizierten Fällen sei der Zellzustand unverändert.

Da in einem Zeitschritt gegebenenfalls mehrere Fälle auf die gleiche Zelle zutreffen können, müssen diese vier Funktionen noch in der richtigen Reihenfolge zusammengesetzt werden. Die Definition einer \emph{Komposition} von Übergangsfunktionen ist noch recht naheliegend:
\begin{align*}
    \circ : Q^{Q \times Q \times Q} \times Q^{Q \times Q \times Q} &\rightarrow Q^{Q \times Q \times Q} \\
    (\delta_1 \circ \delta_2)(q_1,q_2,q_3) &:= \delta_1(q_1,\delta_2(q_1,q_2,q_3),q_3)
\end{align*}
Schwieriger ist dagegen die Festlegung der Reihenfolge. Es ergeben sich folgende Einschränkungen:
\begin{enumerate}
    \item $\pushZiel \leftarrow \pushQuelle$: Wurde gerade ein Zeichen in die Zelle gepusht und dadurch alle Register gefüllt, darf \pushQuelle das letzte Register nicht im gleichen Zeitschritt wieder leeren.
    \item $\popQuelle \leftarrow \popZiel$: Wurde durch \popQuelle die Zahl der gefüllten Zellen von zwei auf eine verringert, darf \popZiel im gleichen Zeitschritt nicht schon ein Zeichen von rechts übernehmen, da für den rechten Nachbarn die Zelle noch nicht unterfüllt erscheint.
    \item $\popQuelle \leftarrow \pushQuelle$: Sieht der rechte Nachbar eine voll gefüllte Zelle und wird das dritte Register übernehmen, darf \pushQuelle dessen Inhalt nicht ins zweite Register schieben.
    \item $\pushZiel \leftarrow \popZiel$: Sieht der rechte Nachbar, dass nur eine Zelle gefüllt ist, und wird sein erstes Register löschen, muss \popZiel dieses übernehmen, bevor \pushZiel die Zahl der gefüllten Register ggf. auf 2 erhöht.
\end{enumerate}

\begin{figure}[h]
    \centering
    \begin{tikzpicture}[>=stealth]
        \node (pushZiel) { \pushZiel };
        \node (popQuelle) [below=of pushZiel] { \popQuelle };
        \node (pushQuelle) [right=of pushZiel] { \pushQuelle };
        \node (popZiel) [right=of popQuelle] { \popZiel };
        \draw[->] (pushQuelle) edge (pushZiel) edge (popQuelle)
                  (popZiel) edge (pushZiel) edge (popQuelle);
    \end{tikzpicture}
    \caption{Abhängigkeitsdiagramm der vier Übergangsfunktionen}
\end{figure}
Damit ergibt sich als eine von vier Möglichkeiten beispielhaft
    \[ \delta := \delta_{\pushZiel} \circ \delta_{\popQuelle} \circ \delta_{\popZiel} \circ \delta_{\pushQuelle} \]

Es verbleiben die Übergangsfunktionen des Stackkopfes. Diese können in diesem Fall allerdings leicht aus $\delta$ abgeleitet werden, indem die jeweilige Operation als virtuelle Zelle links vom Kopf betrachtet wird:
\begin{align*}
    \delta_{\push}(a, q_1, q_2) &:= \delta(\ve{b}{b}{a}, q_1, q_2) \\
    \delta_{\pop}(q_1, q_2) &:= \delta(\ve{\epsilon}{\epsilon}{\epsilon}, q_1, q_2) \\
    \delta_{\nop}(q_1, q_2) &:= \delta(\ve{b}{b}{\epsilon}, q_1, q_2) \\
\end{align*}
wobei $b \in \Sigma$ beliebig ist.

\subsection{Ein 2-Band-Stack mit Zeitverlust}

Exakt analog kann auch ein Stack mit nur zwei Bändern entworfen werden. Dieser hält wieder das letzte Band leer, also nur das erste Band gefüllt, und die vier Teilüberführungsfunktionen ergeben sich aus den angepassten Regeln
\begin{enumerate}
    \item Eine Zelle mit zwei vollen Registern gibt den Inhalt ihres zweiten Registers an ihren rechten Nachbarn ab, der ihn seinen Registern voranstellt.
        \begin{center}
            \begin{tikzpicture}[scale=0.4,>=stealth]
                \filldraw[lightgray] (0,2) rectangle (1,0);
                \filldraw[lightgray] (1,2) rectangle (2,1);
                \draw (0,0) grid (2,2);
                \begin{scope}[shorten >=0.5mm]
                    \draw[->,thick] (0.5,0.5) -- (1.5,1.5);
                    \draw[->,darkgray] (1.5,1.5) -- (1.5,0.5);
                \end{scope}
            \end{tikzpicture}
        \end{center}
    \item Eine Zelle mit zwei leeren Registern stiehlt das erste Register ihres rechten Nachbarn, falls gefüllt.
        \begin{center}
            \begin{tikzpicture}[scale=0.4,>=stealth]
                \filldraw[lightgray] (1,2) rectangle (2,1);
                \draw (0,0) grid (2,2);
                \begin{scope}[shorten >=0.5mm]
                    \draw[->,thick] (1.5,1.5) -- (0.5,1.5);
                \end{scope}
            \end{tikzpicture}
        \end{center}
\end{enumerate}
Da die Symbole im Ruhezustand nun allerdings linear im ersten Band angeordnet sind, kann die \pop-Operation nicht mehr ohne Zeitverlust bedient werden: $\Delta^*_{\pop}=\Delta_{\pop} \circ \Delta_{\nop}$.
\begin{figure}[h]
    \centering
    \begin{tikzpicture}[every node/.style={anchor=base}]
        \matrix [row sep=1cm,column sep=1cm] {
            \draw[scale=0.4] (0,0) grid (4,2);
            \begin{scope}[scale=0.4,xshift=12,yshift=5]
                \node[anchor=east,yshift=4] at (-1,0.5) { 1. \nop };
                \node at (0,1) {a};
                \node at (1,1) {b};
                \node at (2,1) {c};
            \end{scope}
            &


            \filldraw[scale=0.4,lightgray] (0,2) rectangle (1,1);
            \draw[scale=0.4] (0,0) grid (4,2);
            \begin{scope}[scale=0.4,xshift=12,yshift=5]
                \node[anchor=east,yshift=4] at (-1,0.5) { 2. \pop };
                \node at (1,1) {b};
                \node at (2,1) {c};
            \end{scope}
            \\

            \filldraw[scale=0.4,lightgray] (1,2) rectangle (2,1);
            \draw[scale=0.4] (0,0) grid (4,2);
            \begin{scope}[scale=0.4,xshift=12,yshift=5]
                \node[anchor=east,yshift=4] at (-1,0.5) { 3. \nop };
                \node at (0,1) {b};
                \node at (2,1) {c};
            \end{scope}
            &

            \filldraw[scale=0.4,lightgray] (0,2) rectangle (1,1);
            \filldraw[scale=0.4,lightgray] (2,2) rectangle (3,1);
            \draw[scale=0.4] (0,0) grid (4,2);
            \begin{scope}[scale=0.4,xshift=12,yshift=5]
                \node[anchor=east,yshift=4] at (-1,0.5) { 4. \pop };
                \node at (1,1) {c};
            \end{scope}
            \\
        };
    \end{tikzpicture}
    \caption{Beispielablauf des 2-Band-Stacks}
\end{figure}



\section{Beweis}

Ziel dieses Abschnittes ist es zu beweisen, dass Kutribs 3-Band-Stack ohne Zeitverlust tatsächlich einen minimale Zahl von Bändern besitzt. Dazu braucht es aber zu allererst eine allgemeinere Definition von \emph{Bändern}:
\begin{definition}
    Wir nennen eine Familie von Zellularautomaten $\ca_\Sigma$ \emph{$k$-Band-Automaten}, falls für die jeweilige Zustandsmenge $Q_\Sigma$ gilt:
    \begin{equation}
        \abs{Q_\Sigma} \in \mathcal{O}(\abs{\Sigma}^k) \label{def:ntape}
    \end{equation}
\end{definition}

Die tatsächliche Form der Zustandsmenge des Automaten kann beliebig weit von $k$-Tupeln entfernt sein, aber \eqref{def:ntape} sichert zu, dass wir sie stets auf unsere gewohnte Definition eines $k$-Bandes zurückführen können:

\begin{satz}
    \label{thm:tape-hom}
    Jeder $k$-Band-Automat lässt sich in einen Automaten mit der Zustandsmenge $\zrange{\ceil\alpha} \times \Sigma^k$ überführen, der an seinen Operationen gemessen gleich arbeitet. Dabei ist $\alpha$ die Konstante aus dem $\mathcal{O}$-Kalkül in \eqref{def:ntape}.

    \begin{beweis}
        Die Aussage folgt wegen $\abs{Q_\Sigma} \leq \alpha \abs{\Sigma}^k \leq \ceil\alpha \abs{\Sigma}^k$ aus Mächtigkeitsargumenten.
    \end{beweis}
\end{satz}

Dieser Satz sagt allerdings nichts aus über die tatsächliche Speicherung der Inhalte im Zellraum, insbesondere muss ein gepushtes Zeichen $a \in \Sigma$ nicht in dieser Form überhaupt auf einem Band stehen. Wir greifen deswegen auf $\Gamma$ zurück und erweitern es:

\begin{definition}
    Für eine zelluläre Datenstruktur definieren wir den Inhalt der ersten $n$ Zellen als
        \[ \Gamma(c,n) := \Gamma(c) \big|_m \]
    wobei $\Gamma(c) \big|_m$ den Präfix von $\Gamma(c)$ der Länge $m$ bezeichnet und $m$ definiert ist als
    \[ m := \max \{m' \in \N \mid \Gamma(c) \big|_{m'} = \Gamma(c') \big|_{m'} \;\forall c' \text{ valide Konfiguration}, c' \big|_n = c \big|_n \} \]
    also $\Gamma(c,n)$ längster gemeinsamer Inhaltspräfix aller Konfigurationen mit den gleichen ersten $n$ Zellen ist oder anders gesagt der Inhalt, der von allen Zellen $> n$ unabhängig ist.

    Für Automaten mit Zustandsmenge $\Sigma^k$ ist dies normalerweise die Konkatenation des Inhalts der ersten $n$ Zellen.
\end{definition}

\begin{satz}
    \label{thm:locontent}
    Nach Pushen von $2n-3$ Zeichen gibt es einen Zeitpunkt $t_0$, sodass $\abs{\Gamma(c,n)} \geq 2n-3$.
    \begin{beweis}
        Welchen Präfix von $\Gamma(c)$ kann eine Konfiguration $c'$ mit $c' \big|_n = c \big|_n$ verändern? Das Signal des ersten Pops erreicht $c'(n+1) \neq c(n+1)$ zum Zeitpunkt $t_0+n$. Damit könnte $c'^{t_0+n-1+n-2}(2)$ beeinflusst werden, also die $2n-3$-te Pop-Operation.
    \end{beweis}
\end{satz}

\begin{satz}
    \label{cor:spacesize}
    Für $k$-Band-Automaten gibt es für alle $n \in \N, l > kn$ ein Alphabet $\Sigma$ und ein Wort $w \in \Sigma^l$, nach dessen Pushen gilt:
    \[ \abs{\Gamma(c,n)} \leq kn \]
    \begin{beweis}
        Angenommen, es gälte für ein $n$ und $l$ und alle Alphabete $\Sigma$ und Konfigurationen $c$ $\abs{\Gamma(c,n)} > kn$, also $\abs{\bild{\Gamma(\cdot,n)}} \geq \abs{\Sigma}^{kn+1}$.
        Es folgt
        \[ \abs\Sigma^{kn+1} < \abs{\bild{\Gamma(\cdot,n)}} \leq \abs{Q^n} \leq (\ceil{c}\abs{\Sigma}^k)^n = \abs{\Sigma}^{kn} \ceil{c}^n \]
        also
        \[ kn + 1 \leq kn + n \log_{\abs\Sigma}(\ceil{c}) \]
        Für $\abs{\Sigma}$ groß genug ergibt sich der Widerspruch.
    \end{beweis}
\end{satz}

\begin{satz}
    Es gibt keinen 2-Band-Automaton mit \pop-Operation ohne Zeitverlust.
    \begin{beweis}
        Wähle $w \in \Sigma^{2n+1}$ nach \thref{cor:spacesize}. Nach Pushen von $w \big|_{2n-3}$ gibt es nach \thref{thm:locontent} $t_0$, sodass $\abs{\Gamma(c,n)} \geq 2n-3$. Pushe dann die letzten vier Buchstaben und es gilt, da $c(n+1)$ nicht beeinflusst worden sein kann, $\abs{\Gamma(c,n)} \geq 2n+1$, im Widerspruch zu \thref{cor:spacesize}.
    \end{beweis}
\end{satz}

\begin{thebibliography}{9}

\bibitem{kutrib08}
    M. Kutrib: \emph{Cellular Automata – A Computational Point of View}, Studies in Computational Intelligence (SCI) 113, 183–227 (2008)

\end{thebibliography}
\end{document}
